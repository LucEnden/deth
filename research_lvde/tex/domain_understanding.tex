\chapter{Domain understanding}

TODO: write a short intro to this chapter in which explaining what pokémon is by giving a brief summery of its history and then explaining where this research links to pokémon. 

\section{Manga, Anime and (card) Game(s)}

TODO: write how pokémon developed over time from a manga to whatever

TODO: add 3 images, a page from a mange, a still from the anime and a card from the card game

TODO: add footnotes to the images with "Figure N: ..."

\section{About the games}

TODO: write how the traditional games developed, not going into detail about spin off games, starting at pokémon red and green up to black and white 2 (it continues but I have no experience with these games, and going into their details is not of added value to this domain understanding)

\section{A quick intro to the Gen. 4 games}

TODO: write that from here on the domain understanding is tailered specificly towards generation 4 as thats what this study is tailered towards.

TODO: write a short intro explaining that starting the game, go trough some dialouge, choose a pokémon after which you have your first battle. Also describe that details about battleing will be done in the definitions chapter.

\section{A perfect information game}

TODO: write that everything about the game is extensivly documented online. paired with that fact, as long as a person has read access to the games state (stored in memory), they can retrieve information not being displayed on screen (like hidden pokémon stats), making pokémon a perfect information game. This knowledge will at first not be utilized during training, as we want the agents to behave preferably like a normal human, which would not have this perfect infromation before hand. But we will not be excluding this knowledge entierly, as it might proof usefull.

\section{What is a Pokémon?}

When stating the word pokémon, people will often times have a intuitive definition of what that means. Usually people will think of their favorite or the most commonly known pokémen, which is often regarded as being Pickachu.

TODO insert image of pickachu and align it to the right

*Figure N: an image of pickachu, often regarded as the most comonly known pokémon as its the mascot of the company*

A more formal, mathimatical definition of what a pokémon is will be required to perform studies on them. This definition will be provided later in this study.