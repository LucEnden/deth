\chapter{Domain Understanding}

This chapter provides a brief introduction to the world of Pokémon, tracing its development from its origins to the focus of this study, and contextualizing its relevance.

\section{Manga, Anime, and (Card) Games}
Pokémon began as a manga series created by Satoshi Tajiri and Ken Sugimori in 1996, accompanying the release of the first video games, Pokémon Red and Green. The franchise quickly expanded into an anime series that became a global phenomenon, featuring Ash Ketchum and his iconic partner Pikachu. Building on its success, the Pokémon Trading Card Game was introduced, offering a new way for fans to interact with the franchise. Over time, Pokémon grew into a multimedia empire spanning video games, films, merchandise, and competitive tournaments, becoming one of the most recognizable brands worldwide.

\section{About the Games}
The core Pokémon games, starting with Red and Green, are role-playing games (RPGs) developed by Game Freak and published by Nintendo. These games revolve around capturing, training, and battling creatures called Pokémon. Players embark on journeys through fictional regions, aiming to defeat gym leaders and ultimately become the Pokémon Champion. Each generation introduced new regions, mechanics, and Pokémon species, enhancing gameplay and strategy.

From the first-generation games Red and Green to fifth-generation games Black and White 2, the franchise evolved significantly. Key innovations included the introduction of new battle mechanics such as abilities, held items, and double battles, as well as enhanced graphics and expanded narratives. This study focuses on the fourth generation, encompassing Diamond, Pearl, Platinum, HeartGold, and SoulSilver, which introduced mechanics like the physical/special split and online play.

\section{A Quick Introduction to the Generation 4 Games}
From this point, the domain understanding will be tailored specifically toward the fourth-generation games, as this study centers on their mechanics and structure.

In these games, players begin their journey by engaging in a brief introduction to the game’s world and controls. After some initial dialogue, they are prompted to choose a starter Pokémon from a set of three. This decision is followed by the player’s first battle, providing an introduction to Pokémon battling mechanics. Details about battles, including their rules and strategies, are elaborated in the Definitions chapter.

\section{A Perfect Information Game}
Pokémon is considered a perfect information game because its mechanics and data are extensively documented online. Moreover, if a player or agent has read access to the game’s state stored in memory, they can retrieve hidden information, such as detailed Pokémon stats or move properties, that is not displayed on the screen.

Initially, this study avoids utilizing such perfect information during training, aiming for agents to mimic human behavior, which typically lacks this prior knowledge. However, perfect information will not be excluded entirely, as it may prove useful in advanced scenarios or further research.

\section{What is a Pokémon?}

When stating the word pokémon, people will often times have a intuitive definition of what that means. Usually people will think of their favorite or the most commonly known pokémen, which is often regarded as being Pickachu. However, in order to perform studies on pokémon, a more formal definition is required.

A more formal, mathimatical definition of what a pokémon is will be required to perform studies on them. This definition will be provided later in this study.