\chapter{The Goal of This Study}

The primary goal of this study is to **create a probabilistic model that learns to play Pokémon Platinum**. This involves leveraging Markov decision processes or similar probabilistic frameworks to model and solve various challenges encountered in the game.

\section{Minimal Viable Product (MVP)}
The minimum objective is to develop a model capable of learning how to engage in battles and achieve victory. The precise definition of "winning" in this context will be refined as the study progresses, incorporating factors such as efficiency, success rate, and adaptability.

\section{Extended Goals}
If time permits, the study will also explore the development of an exploratory model designed to navigate the overworld. This model would aim to traverse the game world efficiently and interact with various game elements, providing a more comprehensive framework for automated gameplay.

By balancing these objectives, the study seeks to demonstrate the feasibility and effectiveness of probabilistic models in addressing the complex decision-making tasks inherent to Pokémon Platinum.

\section{Gradually Increasing the Model's Goal Difficulty}

The difficulty of the model's objectives will be increased incrementally as the agent demonstrates successful training and proficiency at simpler tasks. This step-by-step approach ensures that the model builds a solid foundation of skills before tackling more complex challenges.

For example, initial tasks may focus on basic decision-making, such as selecting effective moves during battles or navigating straightforward overworld paths. Once the model achieves consistent success in these areas, the complexity of the tasks will gradually increase to include:
- Optimizing strategies for tougher battles, such as those involving type advantages or multi-stage opponents.
- Navigating more intricate overworld segments with branching paths or hazards.
- Managing resources like healing items or determining when to retreat to a Pokémon Center.

This progressive approach allows the model to adapt and improve over time, ensuring it is equipped to handle increasingly difficult scenarios. Additionally, this structure provides clear milestones for measuring the model’s success throughout its development.